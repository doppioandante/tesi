\chapter{Lo standard MIDI}

\section{Il protocollo di comunicazione e i messaggi MIDI}
Nel protocollo MIDI le informazioni e gli eventi vengono trasmessi sotto forma di \textit{messaggi}.
Ogni messaggio è composto da una sequenza di byte ordinata, di cui il primo è detto \textit{status byte} e i successivi
vengono detti \textit{data byte}.
Il bit più significativo degli status byte è 1, mentre quello dei data byte è 0, per cui ogni data byte permette
solo valori da 0 a 127. I primi 4 bit dello status byte identificano il tipo di messaggio MIDI, mentre gli ultimi 4 specificano il canale su cui il messaggio ha effetto, per un massimo possibile di 16 canali.

Ai fini del progetto, solamente i messaggi di \textit{note on} e \textit{note off} sul canale '0' verranno gestiti, mentre
ogni altro tipo di messaggio sarà ignorato.
Ogni messaggio di note on viene trasmesso nel momento in cui una certa nota deve essere suonata e contiene due data byte: il primo, il note number, identifica la nota da riprodurre, mentre il secondo, detto velocity, fornisce un'informazione
sull'intensità con cui la nota viene suonata.
Il messaggio note off è analogo, ma con uno status byte differente.


\section{Physical Layer}
Ogni byte nel protocollo MIDI viene trasmesso in un data frame composto da:
\begin{itemize}
	\item \textbf{Start bit}, inizio del byte segnalato dal valore logico basso
	\item
	 La sequenza di 8 bit, in ordine dal meno significativo al più significativo
	\item \textbf{Stop bit}, che indica la fine del byte ed è segnalata dal valore logico alto
\end{itemize}

(aggiungere baud rate e considerazione su assenza di parity bit)
In questo aspetto il protocollo MIDI è simile al protocollo seriale RS-232, con la differenza che i valori logici alti sono segnalati dallo scorrere di una corrente di $\SI{5}{\milli\ampere}$ invece che dalla presenza di un voltaggio positivo.
Il valore logico basso corrisponde all'assenza di corrente.

\section{Il MIDI tuning standard}
Ad ogni \textit{note number} del protocollo MIDI viene associata una frequenza determinata dal MIDI tuning standard (MTS).
Si suddivide ogni ottava in 12 note equidistanziate secondo una progressione geometrica, per cui per ogni nota di frequenza $f$ vale

\[
fs^{12} = 2f 
\]

Da cui si ricava la ragione $s=2^{\frac{1}{12}}$ della progressione geometrica. Moltiplicando una frequenza per $s$ si ottiene la frequenza della nota successiva.
Fissando convenzionalmente la frequenza della nota numero 69 a $ \SI{440}{\hertz}$, si ottiene la corrispondenza tra
il note number $i$ e la frequenza $f$ associata:

\[
f = \SI{440}{\hertz} \cdot s^{69-i}
\]


