\DeclarePairedDelimiter\floor{\Big \lfloor}{\Big \rfloor}

\chapter{Sintesi digitale del segnale}
\section{Sintesi digitale diretta}
La sintesi digitale diretta (Direct Digital Synthesis, Direct Digital Frequency Synthesis, DDS) permette di generare un segnale periodico di frequenza programmabile attraverso un sistema digitale.

Sia $x(\phi)$ un segnale periodico di periodo $2\pi$ e tale che $|x(\phi)| \le 1$.
In termini formali, si vuole generare un segnale $s(t)=Ax(2\pi ft)$ periodico di periodo $T=\frac{1}{f}$, di ampiezza $A$, dove
\begin{equation}
\label{eq:phase}
\phi(t) = 2\pi f t
\end{equation}
viene detta \textbf{fase} del segnale, ed è inizialmente nulla.

Si noti che le ipotesi fatte sul segnale $x(\phi)$ non sono restrittive in quanto
un qualsiasi segnale $y(z)$ periodico di periodio $T_y$ e tale che $|y(z)| \le y_{max}$ può
essere ricondotto ad un segnale dello stesso tipo di $x$ operando la trasformazione
\[
x(\phi)=\frac{1}{y_{max}}y \Big ( \frac{T_y}{2\pi}\phi \Big )
\]

Un sistema di DDS si compone di due blocchi: il \textbf{phase accumulator} (PA), responsabile di generare la fase del segnale a una determinata frequenza, e il \textbf{phase to amplitude converter} (PAC) che, a partire dalla fase, ottiene l'ampiezza del segnale.

In un sistema a campioni, il PAC si avvale di una ROM di $2^{\code{M}}$ parole contenente il segnale campionato a \code{b} bit.
Altri modi di implementare il PAC sono algoritmici, come ad esempio l'algoritmo CORDIC
che viene impiegato per generare segnali di tipo sinusoidale.

Il \textit{phase accumulator} è un registro a $\code{N}$ bit il cui valore rappresenta la frazione del periodo $[0, 2\pi]$ del segnale
$x(\phi)$ in notazione
fixed point. Ad ogni fronte di salita del clock il valore $\code{F}$ del registro viene incrementato di un valore pari alla \textit{frequency tuning word},
sempre di $\code{N}$ bit, che è in relazione biunivoca con la frequenza $f$ desiderata.

Nella sintesi digitale, la fase al tempo $t$ si ottiene dal registro di fase secondo quando descritto prima:
\begin{equation}
\label{eq:phasereg}
\phi = 2\pi\frac{\code{F}}{2^\code{N}}
\end{equation}

Essendo \code{F} a precisione finita, si avrà un errore sulla precisione della fase del segnale.

Sommando la frequency tuning word al valore del registro di fase \code{F} si ottiene il nuovo valore \code{F'}
e poiché l'addizione avviene ad ogni periodo di clock $T_{clk}$, i valori della fase del segnale corrispondenti sono
$\phi(t)$ e $\phi(t + T_{clk})$.
Si può quindi ricavare la \code{FTW} utilizzando le
\cref{eq:phase,eq:phasereg}, imponendo:
\[
\code{FTW} = \code{F'} - \code{F} = 
\floor{\frac{2^\code{N}}{2\pi}\phi(t+T_{clk}) - \frac{2^\code{N}}{2\pi}\phi(t)}
= \floor{2^\code{N} f T_{clk}} = \floor{2^\code{N} \frac{f}{f_{clk}}}
\]
dove $\floor{\cdot}$ indica la parte intera. \\
La frequenza effettiva $f_{o}$ del segnale risultante dalla DDS è dunque
\begin{equation}
\label{eq:actualphase}
f_{o} = \frac{\phi(T_{clk}) - \phi(0)}{2\pi T_{clk}} = \frac{2\pi \code{FTW}}{2^\code{N} 2 \pi T_{clk}} = \frac{\code{FTW}}{2^N} f_{clk}
\end{equation}
in generale diversa dalla $f$ desiderata per via dell'errore di troncamento della fase.
Tuttavia, scegliendo un numero di bit sufficiente per il registro di fase, si riesce
ad ottenere una precisione molto elevata: usando la \cref{eq:actualphase}
e imponendo $\code{FTW}=1$ e $N=32$, si ricava come risoluzione del registro di fase 
un valore di circa $\SI{0.02}{\hertz}$.

In generale non è necessario che il segnale da rappresentare sia a fase
iniziale nulla ed è facile rappresentare un segnale con uno sfasamento
iniziale arbirario usando il registro di fase.
Cambiando la FTW durante il processo di DDS è possibile cambiare
quasi istantaneamente la frequenza del segnale, cosa invece difficile
da ottenere con tecniche analogiche.

Il registro di fase si può considerare come una versione quantizzata della
fase reale del segnale $\phi(k T_{clk})$, a meno di multipli di $2\pi$.
La fase è quindi un segnale avente la stessa frequenza $f$ rappresentata dalla frequency
tuning word impiegata e campionato alla fequenza di clock, per cui vale il teorema
del campionamento di Shannon:
\[
   2f < f_{clk} \implies f < \frac{f_{clk}}{2}
\]
È quindi impossibile generare segnali aventi frequenza maggiore della metà della
frequenza di clock, condizione ampiamente rispettata per i segnali audio:
l'udito umano è capace di percepire frequenza fino a circa $\SI{21}{\kilo\hertz}$, 
per cui, dal teorema del campionamento, è necessaria una frequenza di campionamento $f_s$ di almeno $\SI{42000}{\hertz}$.
Il PAC genera un nuovo valore alla frequenza di campionamento scelta, usando
una ROM contenente i campioni, a partire dal valore del registro di fase.

\section{Formato della ROM dei campioni}
\label{sec:quantizationrom}
La ROM viene generata quantizzando il segnale $Ax(\phi)$ nell'intervallo
$[0, 2\pi]$ e $0 < A < 1$.
Sebbene la precisione del registro di fase possa essere molto elevata, non è
possibile quantizzare il segnale per ogni possibile valore del registro di fase:
ad esempio, un registro di fase a 32 bit comporterebbe il campionamento di $2^{32}$ valori,
e considerando una risoluzione di $11$ bit si dovrebbero impiegare $11\cdot2^{32}$bit,
corrispondenti a \SI{5.5}{GiB}, che è una quantità di memoria proibitiva.
Si campiona dunque il segnale $x(\phi)$ in $2^\code{M}$ punti, con $\code{M} < \code{N}$
e si quantizza uniformemente con una precisione di $\code{b}$ bit.
All'indirizzo di memoria $\code{I}$ si trova il campione corrispondente a
$x(2\pi \frac{\code{I}}{2^\code{M}})$ il cui valore quantizzato $\code{Q}$ a $b$ bit è
\begin{equation}
\label{eq:quantization}
\code{Q} = \floor{2^\code{b} Ax(2\pi \frac{\code{I}}{2^\code{M}})}
\end{equation}
I valori quantizzati vengono rappresentati in notazione binaria in complemento a due,
permettendo così l'uso dell'aritmetica con segno nella fase di mixing del sintetizzatore.
L'indirizzamento a partire dal registro di fase si ottiene prendeno i primi \code{M}
bit del registro come indirizzo di memoria della ROM: si ha quindi un errore aggiuntivo
nella DDS dovuto alla limitatezza della memoria dei campioni.

\section{Conversione D/A attraverso PWM}
\label{sec:dapwm}
Per portare il segnale discreto generato attraverso la DDS al dominio analogico si fa uso di una tecnica di modulazione detta \textbf{pulse-width modulation} (PWM).

Per comprendere il funzionamento della PWM si consideri il caso in cui si voglia rappresentare un segnale costante $x(t) \equiv x_0$ con $x_0 \in [0, x_{max}]$.
L'ipotesi di un segnale $x(t)$ positivo, sebbene non generale, è sufficiente
a caratterizzare il segnale PWM processato dal filtro passa-basso della scheda
Nexys, essendo questo un segnale a voltaggio maggiore o uguale a zero.
Si dovrà quindi ricondurre il segnale campionato nella ROM, nel range $[-1,1]$,
ad un segnale nel range $[0,x_{max}]$ con opportune operazioni di scala e
traslazione.

Sia quindi $x_{pwm}(t)$ un'onda quadra di ampiezza $x_{max}$, di periodo $T$ e di duty cycle $d=\frac{x0}{x_{max}}$, $d \le 1$:
\[
x_{pwm}(t) = \begin{cases}
	x_{max} & \text{se}\ kT \le t < (k+d)T \\
	0 & \text{se}\ (k+d)T \le t < (k+1)T
\end{cases}
\] 
con $k$ intero.

Questo segnale è sviluppabile in serie di Fourier, ed essendo un'onda quadra
conterrà i multipli dispari dell'armonica fondamentale di frequenza $\frac{2\pi}{T}$.
Il coefficiente $a_0$ della serie è inoltre pari alla media del segnale sul periodo, ossia
\[
a_0 = x_{max} \cdot d = x_0
\]

È possibile quindi ricostruire il segnale $x(t)\equiv x_0$ a partire da $x_{pwm}(t)$ passandolo attraverso un filtro passa-basso che rimuova tutte le armoniche di frequenza
più alta di $a_0$.

Nel caso in esame $x(t)$ non sarà costante, ma sarà variabile e rappresenterà la forma d'onda generata dal sistema di sintesi digitale diretta.

Al segnale costante verrà quindi sostituito il segnale a tempo discreto $x(kT), k \in Z$ generato alla frequenza di campionamento $f_s = \frac{1}{T}$ e con valori $0 \leq x(kT) \leq x_{max}$, e in prima approssimazione il segnale $x_{pwm}$ corrispondente sarà
\[
x_{pwm}(t) = \begin{cases}
	x_{max} & \text{se}\ kT \le t < (k + d_k)T \\
	0 & \text{se}\ (k+d_k)T \le t < (k+1)T
\end{cases}
\]
dove $(d_k)_{k \in \mathbb{N}}$ è una sequenza di duty cycle non più costanti ma \textbf{modulati} e dati da
\[
d_k = \frac{x(kT)}{x_{max}}
\]

Tuttavia il segnale PWM è generato digitalmente, quindi supponendo per semplicità $T=1/f_s$ multiplo del periodo di clock $1/f_{clk}$, in un singolo periodo
di campionamento si avrà una successione di $N=\frac{f_{clk}}{f_s}$ impulsi pari a $x_{max}$ (corrispondente ad un valore logico alto) o $0$ (corrispondente a un valore logico basso).
Poiché al fine della modulazione PWM conta solo il valore medio sul periodo, l'ordine degli impulsi non cambia il risultato, anche se cambia la distribuzione delle armoniche nello spettro del segnale.
È quindi possibile rappresentare un numero finito di valori in uscita pari a $N$, e quindi i campioni forniti potranno al massimo avere una risoluzione di $\log_2{N}$ bit.

La modulazione PWM non è lineare e introduce quindi distorsione nel segnale ricostruito.
