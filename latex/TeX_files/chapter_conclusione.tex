\chapter{Conclusione e possibili sviluppi}

Il sintetizzatore progettato, sebbene funzionante, presenta performance
di SNR non ottimali, dovute alle limitazioni intriseche della pulse-width 
modulation.
Sarebbe possibile bypassare l'uscita audio monofonica della scheda
e utilizzare un device esterno, connesso attraverso le porte PMOD, 
per fornire una qualità audio migliore.
Due esempi sono i componenti PMOD I2S2 e il PMOD AMP2 della Digilent.

La polifonia, essendo implementata con un semplice sommatore in fase 
di mixing, è soggetta ad overflow quando la somma dei campioni supera
la precisione permessa dai \code{b} bits di quantizzazione.
In campo audio questo fenomento viene detto \textit{clipping}.
Di fatto, essendo l'ampiezza del segnale campionato pari a poco meno
della metà del valore massimo rappresentabile, la polifonia è limitata
a due voci, oltre le quali il segnale presenta una degradazione
notevole.

Far rientrare il range del segnale nei limiti con un divisore risolverebbe
il clipping ma degraderebbe la qualità del segnale e causerebbe
sbalzi di volume udibili nel passaggio da una singola nota  a più note
suonate contemporaneamente.

Tecniche più avanzate di compressione dinamica del segnale 
sono molto complesse e al di fuori dello scopo di questa tesi.  

I messaggi MIDI contengono anche l'informazione relativa alla forza
di pressione della nota, detta velocity, come descritto nel \cref{chap:midi}.
Si potrebbe estendere il sintetizzatore per rispettare questa informazione
in due modi differenti:
\begin{itemize}
  \item Si stabiliscono dei range di velocity meno granulari  e per ognuno
        di questi si accede a una rom diversa creata impostando una diversa 
        ampiezza alla forma d'onda
  \item Si stabiliscono sempre dei range di velocity, ma lo scaling dei
        campioni avviene con un divisore
\end{itemize}
Il primo metodo produrrebbe risultati più precisi numericamente ma al
prezzo di una notevole quantità di memoria in più, mentre il secondo
otterrebbe una qualità peggiore ma con un'occupazione minore delle
risorse del FPGA.

Per mancanza di tempo il sintetizzatore è stato operato attraverso un 
computer per il testing, ma sarebbe possibile connettere la scheda
direttamente utilizzando una porta PMOD generica e un circuito esterno
(anche su breadboard) a cui connettere il jack MIDI.
