\chapter{Testing e analisi delle prestazioni}
Il testing del progetto sulla scheda Nexys4 DDR è stato condotto
attraverso la porta seriale della scheda e mediante l'uso di uno
script Python che automatizzasse la riproduzione di un file MIDI 
(si veda \code{scripts/play\_midi\_file.py} (aggiungere citazione al codice).

Per effettuare il testing funzionale delle varie componenti del progetto
si è fatto uso di \textit{testbenches} scritti in VHDL, che fornissero
un input prefissato a un certo componente per verificarne la risposta,
utilizzando sempre GHDL.
In particolare il testbench descritto nel file
 \code{vhdl/mono\_synth\_engine\_tb.vhd} provvede a testare l'intera
architettura riproducendo una singola nota.
L'uscita del blocco \code{pwm\_encoder} viene quindi salvata ad ogni
ciclo di clock su un file txt, che viene in seguito processato
dallo script \code{scripts/plot\_pwm\_output.py} che lo filtra
digitalmente e visualizza il segnale ricostruito.

Data la complessità computazionale di una simulazione, il testbench
di prova dell'intero progetto può impiegare fino a diversi minuti
per generare un output che realmente avrebbe durata nell'ordine della
decina di millisecondi.

\section{Qualità del segnale sintetizzato}

