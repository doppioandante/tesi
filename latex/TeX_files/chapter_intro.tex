\chapter{Descrizione e funzionamento di un sintetizzatore musicale}
Un sintetizzatore musicale permette la riproduzione di suoni di carattere, intensità e frequenza impostabili dal musicista.
L'utilizzo dei sintetizzatori come strumento, sia in studio che dal vivo, si afferma negli anni settanta grazie all'invenzione del \textit{Minimoog}, meno costoso e voluminoso dei sintetizzari a muro degli anni sessanta.
Il \textit{Minimoog} era un sintetizzatore analogico e monofonico, cioè capace di riprodurre solamente una nota alla volta, al contrario dei moderni sintetizzatori detti \textit{polifonici}.
Al giorno d'oggi i sintetizzatori sono per la maggior parte implementati in maniera digitale, e permettono l'utilizzo di una varietà di tecniche di sintesi (aggiungere)
In questa tesi ci occuperemo di un semplice sintetizzatore polifonico, capace di riprodurre forme d'onda campionate e controllabile attraverso il protocollo MIDI, ampiamente diffuso nell'industria musicale.
Il progetto è stato implementato sulla scheda Nexys 4 DDR della Digilent, dotata di una FPGA (aggiungere), di una porta seriale usata per il testing, e di un uscita mini-jack monofonica.

\section{La catena del segnale del sintetizzatore}
L'input di un sintetizzatore viene dato dal musicista, attraverso uno strumento fisico. Lo strumento invia al sintetizzatore un segnale di controllo che specifica quali note suonare, a che intensità, quali sono i parametri dello strumento ecc. A tale scopo è universalmente impiegato lo standard MIDI, che specifica sia il protocollo di comunicazione che le caratteristiche elettriche della connessione fisica.
Il sintetizzatore si occuperà di riprodurre i suoni desiderati. Due caratteristiche fondamentali del suono riprodotto sono la sua frequenza e il suo timbro. Per variare queste due componenti sono disponibili una varietà di tecniche.
In questo progetto ci si avvale della \textit{Direct Digital Frequency Synthesis} per la generazione di una frequenza determinata, mentre si farà uso di una ROM contenente la forma d'onda campionata per ottenere il timbro desiderato.
La riproduzione di più note contemporanamente si ottiene attraverso una fase di \textit{mixing}, per cui le forme d'onda delle note desiderate vengono sovrapposte.
Alla fine della catena è presente l'uscita audio, che nel caso della Nexys 4 è un (aggiungere) a cui verrà fornito il segnale opportunamente codificato in pulse-width modulation (?).

(aggiungere figura)

(fare sottoparagrafo di titolo delay?)
Il tempo che intercorre tra la sollecitazione dello strumento e l'emissione del suono corrispondente viene chiamato ritardo di propagazione o \textit{delay}.
Il feedback uditivo è essenziale ai fini di una performance musicale, per cui il ritardo introdotto dal sintetizzatore
dovrebbe essere il più basso possibile, con un limite indicativo di \SI{5}{\milli\second}.
Allo stesso tempo, il tempo di ritardo non può essere inferiore al tempo di produzione di un campione (??? va spiegato meglio il funzionamento del sintetizzatore prima di poterne parlare).
^^ spostare in capitolo su generazione del segnale

