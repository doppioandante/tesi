\chapter{La scheda Nexys4 DDR}

L'hardware utilizzato per il progetto consiste in una scheda Nexys4 DDR.

Il nucleo della scheda è costituito da un FPGA Xilinx Artix 7 XC7A100T.
Attorno ad esso sono collegati vari connettori, sensori e pulsanti.

Per la realizzazione del sintetizzatore si farà uso dei seguenti componenti:
\begin{itemize}
    \item \textbf{Bridge USB-UART} Il bridge USB<->UART permette la lettura
          dei messaggi MIDI, inviati dal computer attraverso la porta seriale
    \item \textbf{Mono Audio Output} Un connettore audio monofonico di tipo minijack 
          viene usato per emettere il suono.
    \item \textbf{Clock Crystal} Il cristallo che fornisce un clock di \SI{100}{\mega\hertz}
          al FPGA e alla scheda.
\end{itemize}

L'output fornito dal jack audio viene generato attraverso un convertitore analogico/digitale,
costituito da un filtro passa-basso a cui viene fornito in ingresso il segnale
da generare in codifica PWM.

Il filtro passabasso è un filtro di Butterworth del quarto ordine, (Sallen-Key?)
Descrizione di almeno la risposta in frequenza


\section{FPGA}
Field-programmable Gate Array (FPGA) è un circuito digitale integrato
capace di implementare le più svariate funzioni in modo programmatico:
può essere utilizzato per la fase di prototipazione di Application-specific
Integrated Circuits (ASIC), oppure per sostituirli nel caso i costi
di integrare il FPGA nel progetto siano minori dei costi di sviluppo
e realizzazione dell'ASIC (che di solito sono più vantaggiosi per volumi
di produzione molto grandi).
Un FPGA, data la sua natura generale, offre performance inferiori sia
in consumi che in prestazioni rispetto ad un ASIC, ma ha come vantaggio
la riprogrammabilità (anche una volta integrato nel progetto), che
è invece impossibile con un circuito integrato specifico. 
Un altro uso che si va affermando negli ultimi anni è l'utilizzo dei
FPGA come acceleratori software, ad esempio per applicazioni di 
trading finanziaro e intelligenza artificiale. (citare?)

\subsection{Struttura di un FPGA}
Un Field-programmable Gate Array (FPGA) è un circuito digitale integrato
avente una struttura regolare, e contente:
\begin{itemize}
    \item \textbf{configurable logic blocks}: in breve \textbf{CLB}, blocchi
            logici configurabili che permettono di realizzare funzioni
            logiche arbitrarie
    \item \textbf{configurable routing}: i blocchi logici sono connessi
          attraverso delle connessioni a loro volta configurabili,
          rendendo possibile creare funzioni logiche a un maggior numero
          di ingressi e più complesse
\end{itemize}

Oltre a queste due componenti fondamentali, un FPGA contiente altri
componenti necessari al suo funzionamento e interfacciamento con
circuiti esterni, tra cui:

\begin{itemize}
    \item \textbf{I/O blocks}: blocchi di input e output,
           permettono al FPGA di interfacciarsi con l'esterno
    \item \textbf{Dedicated blocks}: per rendere più efficiente
           l'implementazione di operazioni comuni, esistono vari blocchi
           non configurabili che eseguono funzioni specifiche
    \item \textbf{Memory blocks}: è possibile trovare anche blocchi di
           memoria all'interno di un FPGA
    \item \textbf{Clock Management Tiles e Clock Routing}: circuiteria necessaria per 
           la gestione e propagazione del segnale di clock
\end{itemize}

Il FPGA utilizzato dalla scheda Nexys 4 DDR è uno Xilinx Artix 7 XC7A100T.
Esso contiene 15850 logic slices (i blocchi elementare di un CLB) e sei
 clock management tiles (CMT) con phase-locked loop.
Inoltre, di importanza per il progetto trattato, sono i 240 DSP slices, 
componenti dedicati che rendono più efficiente
l'implementazione dei sommatori, e i 4860 Kbits di Block Ram (BRAM),
che verranno impiegati in fase di sintesi per memorizzare sia la forma
d'onda campionata che le frequency tuning words impiegate nella digital
direct synthesis.

\subsection{Struttura di un CLB}

\section{Programmazione di un FPGA}
La programmazione di un FPGA avviene a livello fisico attraverso
la scrittura di una SRAM che indica come vanno configurati i blocchi
logici e le interconnessioni presenti nell'IC.

A livello di sviluppo si utilizzano linguaggi di descrizione dell'hardware
(HDL) come VHDL e Verilog, che nascondono in gran parte la complessità
dell'FPGA.

Il codice sorgente viene processato da un programma di sintesi che compie
all'incirca le seguenti operazioni:

\begin{itemize}
    \item \textbf{Creazione del netlist}: il codice viene convertito in una
          netlist composta da componenti ad alto livello (come sommatori,
          multiplexer, registri ecc..) che tuttavia è ancora non combacia
          con alla struttura fisica del FPGA
    \item \textbf{Place and Route}: la netlist viene mappata fisicamente sui
          componenti del FPGA
    \item \textbf{Bitstream Generation}: Il risultato del Place and Route viene
          convertito in una serie di bit che vengono poi scritti sulla SRAM
          del FPGA in fase di programmazione
\end{itemize}

\section{Vincoli temporali}
In fase di progetto e di sviluppo il design può essere simulato e analizzato
ben prima di passare alla programmazione della scheda.
Un passo particolarmente importante è l'analisi dei tempi di propagazione:
successivamente al place and route è possibile calcolare il tempo di propagazione
del segnale attraverso i circuiti del FPGA. Nel caso in cui siano presenti
delle route con un tempo di propagazione troppo lungo, sarà necessario
intervenire a livello del design o addirittura manualmente nel posizionamento
dei blocchi sulla FPGA.
Questo passo viene detto soddisfacimento dei vincoli temporali (timing constraints).

Il periodo di clock utilizzato nel progetto, e il minimo possibile sulla
scheda Nexys, è di $T_{clk} = \SI{10}{\nano\second}$.
In un collegamento tra due elementi sequenziali (ad esempio flip-flop)
il tempo di propagazione dell'uscita del primo elemento all'entrata del
secondo elemento non può quindi superare il periodo di clock.
Ordinando tutte le route per tempo di propagazione decrescente, e
considerando quella avente tempo $t_{max}$ di propagazione maggiore, si definisce
il \textbf{Worse Negative Slack} (WNS) come:

\[
\textrm{WNS} = T_{clk} - t_{max}
\] 

Quando questo diventa negativo vuol dire che i timing constraints non
sono rispettati. 
